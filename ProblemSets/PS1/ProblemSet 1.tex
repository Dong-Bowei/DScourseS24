\documentclass{article}
\usepackage{graphicx} % Required for inserting images

\title{Assignment 1}
\author{Bowei Dong }
\date{}

\begin{document}

\maketitle

\section*{Summary}

My name is Bowei/Bo/BoBo (\textit{he/him}). 

I've enrolled in the DS class for several reasons. Firstly, I aim to diversify my portfolio, preparing myself for both the academia and industry job markets. Secondly, leveraging my background in statistics, I intend to enhance my quantitative skills by revisiting and reinforcing previously acquired knowledge. Additionally, my goal is to formulate one or several potential job market paper ideas during the course. Lastly, based on positive feedback from my cohort and those preceding it, I've chosen this class due to its excellent reputation, and I particularly appreciate Tyler's teaching style in his classes.

I have multiple ideas for this class. Primarily, I'm interested in exploring people's preferences regarding climatic features and how they influence location choices, aligning with the primary focus of my PhD research—internal migration. The significance of internal migration resonates with me, as we all face the need to make decisions about it at various points in our lives. Aspiring to be an urban economist, I'm hoping to become an expert in my research area. Furthermore, I am open to combining urban economics with other fields, such as health and sport economics, to enrich the interdisciplinary aspects of my research.

The goal of taking this class is to delve into a new idea that could potentially become the foundation for my PhD job market paper. Thinking about my plans after graduation fills me with excitement and anticipation. If I choose the academic path, my aspiration is to teach at an R2 university (hopefully somewhere colder, with higher humidity, and heavier snow). On the other hand, if I venture into the industry, I am keen on working as an economist at Goldman Sachs, I know it sounds very challenging, but Batman once said "Everything is impossible until someone does it"!

\section*{Equation}
\begin{equation}
    a^2+b^2=c^2
\end{equation}




\end{document}
